\documentclass{article}

% if you need to pass options to natbib, use, e.g.:
% \PassOptionsToPackage{numbers, compress}{natbib}
% before loading nips_2017
%
% to avoid loading the natbib package, add option nonatbib:
% \usepackage[nonatbib]{nips_2017}
\PassOptionsToPackage{numbers, compress}{natbib}
\usepackage{nips_2017}

% to compile a camera-ready version, add the [final] option, e.g.:
% \usepackage[final]{nips_2017}

\usepackage[utf8]{inputenc} % allow utf-8 input
\usepackage[T1]{fontenc}    % use 8-bit T1 fonts
\usepackage{hyperref}       % hyperlinks
\usepackage{url}            % simple URL typesetting
\usepackage{booktabs}       % professional-quality tables
\usepackage{amsfonts}       % blackboard math symbols
\usepackage{nicefrac}       % compact symbols for 1/2, etc.
\usepackage{microtype}      % microtypography
\bibliographystyle{unsrtnat}

\title{Bayesian Inference of Plasma Diffusion Parameters: An LSSVM based approach}

% The \author macro works with any number of authors. There are two
% commands used to separate the names and addresses of multiple
% authors: \And and \AND.
%
% Using \And between authors leaves it to LaTeX to determine where to
% break the lines. Using \AND forces a line break at that point. So,
% if LaTeX puts 3 of 4 authors names on the first line, and the last
% on the second line, try using \AND instead of \And before the third
% author name.

\author{
  Mandar H.~Chandorkar\thanks{Use footnote for providing further
    information about author (webpage, alternative
    address)---\emph{not} for acknowledging funding agencies.} \\
  Multiscale Dynamics\\
  Centrum Wiskunde Informatica\\
  Amsterdam 1098XG, the Netherlands\\
  \texttt{mandar.chandorkar@cwi.nl} \\
  %% examples of more authors
  %% \And
  %% Coauthor \\
  %% Affiliation \\
  %% Address \\
  %% \texttt{email} \\
  %% \AND
  %% Coauthor \\
  %% Affiliation \\
  %% Address \\
  %% \texttt{email} \\
  %% \And
  %% Coauthor \\
  %% Affiliation \\
  %% Address \\
  %% \texttt{email} \\
  %% \And
  %% Coauthor \\
  %% Affiliation \\
  %% Address \\
  %% \texttt{email} \\
}

\begin{document}
% \nipsfinalcopy is no longer used

\maketitle

\begin{abstract}
  The abstract paragraph should be indented \nicefrac{1}{2}~inch
  (3~picas) on both the left- and right-hand margins. Use 10~point
  type, with a vertical spacing (leading) of 11~points.  The word
  \textbf{Abstract} must be centered, bold, and in point size 12. Two
  line spaces precede the abstract. The abstract must be limited to
  one paragraph.
\end{abstract}

\section{Introduction}

\section{Methodology}

Bayesian inference involves specifying the following components.
\begin{enumerate}
\item Prior distribution over system parameters.
\item Quantification of likelihood over observations.
\item Procedure for performing inference, i.e. exact inference, \emph{Markov Chain
    Monte Carlo}, \emph{Variational methods}, etc.
\end{enumerate}

Performing Bayesian inference over parameters of physical systems,
involves synthesizing pre-exsiting knowledge of the physical system in
question i.e. the \emph{partial differential equation} (PDE), with
statistical techniques. The aim of such an exercise is often the
quantification of uncertainty over system parameters from an often
sparse set of observations which are quantities of interest in the
physical system. 


\subsection{Physical System: Plasma Diffusion}

The radial diffusion system is a simplified one-dimensional version of
the \emph{Fokker-Plank} equation. It tracks the time evolution of the
\emph{phase space density} of particles, $f$ which is goverened by the
differential equation \ref{eq:raddiffusion} known as \emph{radial
  diffusion} \citep{JGRA:JGRA9345}.

\begin{equation}\label{eq:raddiffusion}
  \frac{\partial{f}}{\partial{t}} = l^2 \frac{\partial}{\partial{l}}\left( \frac{\kappa(l,
      t)}{l^{2}} \frac{\partial{f}}{\partial{l}} \right) - \lambda(l,
  t) f +  Q(l, t)
\end{equation}

The \emph{phase space density}, $f$, is a function of the spatial
coordinate $l$ which is also known as the \emph{Roederer} $L^*$ or
\emph{L-shell} in magnetospheric physics \citep{Roederer1970}.

The key quantities in the system above are as follows.

\begin{enumerate}
\item $f$: The density of particles as a function of space $l$ and
  time $t$.
\item $\kappa(l, t)$: Diffusion field.
\item $\lambda(l, t)$: Loss rate, this is a non-negative
  quantitiy which indicates how quickly particles are lost from the
  radiation belts.
\item $Q(l, t)$: Particle injection rate.
\end{enumerate}



\subsubsection*{Diffusion Parameters}

To solve the radial diffusion system \ref{eq:raddiffusion}, the
quantities $\kappa(l, t)$, $\lambda(l, t)$ and $Q(l, t)$ need to be
specified. It is a common practice (see \citet{GRL:GRL10762},
\citet{JGRA:JGRA15067}, \citet{JGRA:JGRA18021} and
\citet{GRL:GRL22815}) to parameterise the diffusion field
$\kappa$ and loss rate $\lambda$ in the following manner.

\begin{equation}
  \kappa(l,t), \lambda(l, t) \sim \alpha l^{\beta} 10^{b Kp(t)}
\end{equation}

The quantities $\alpha$, $\beta$ and $b$ are parameters which define
the diffusion field and loss rate while the quantity $Kp(t)$ is known
as the Kp index, a measured quantity which stands as a proxy for the
global geomagnetic activity \cite{BartelsKp}.

\subsection{Model Formulation}


\subsubsection*{Approximate Forward Model}

\subsubsection*{Quantifying Observation Likelihood}

\subsection{Inference}

\section{Results}

\subsubsection*{Acknowledgments}

Use unnumbered third level headings for the acknowledgments. All
acknowledgments go at the end of the paper. Do not include
acknowledgments in the anonymized submission, only in the final paper.

\section*{References}

References follow the acknowledgments. Use unnumbered first-level
heading for the references. Any choice of citation style is acceptable
as long as you are consistent. It is permissible to reduce the font
size to \verb+small+ (9 point) when listing the references. {\bf
  Remember that you can go over 8 pages as long as the subsequent ones contain
  \emph{only} cited references.}
\medskip

\small

\bibliography{references}
\end{document}
